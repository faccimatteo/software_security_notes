\section{Side-Channel Attacks}

\subsection{What is a side-channel attack?}

When designing cryptographic computations it is given as an assumption that the internal parts and implementations is not observed by a malicious entity.

In reality, this scenario is unrealistic as an attacker will always try to find a way to discover what is "under the hood" and potentially break this cryptographic computations by understanding its cryptographic scheme. 

The first time side-channel attacks was used it exploited the different click-sound of the machine to understand the passphrase inserted. 

The efficiency of side-channel attacks is given as there is a correlation between the physical measurements taken during computation (different parameters can be taken into consideration, as power consumption, computing time, EMF radiation, etc.) and internal state of the device itself. 

\subsection{Models of side channel attacks}

\begin{itemize}
    \item traditional security model: what is exploited is the mathematical specification of the protocol
    \begin{figure}
        \centering
        \includegraphics[scale=0.7]{img/traditional_security_model.png}
        \caption{Traditional security model.}
        \label{fig:tra_sec_mod}
    \end{figure}
    \item security model including side-channel: adversary may be able to monitor the power consumed or the electromagnetic
    radiation emitted by a smart card while it performs private-key operations such as
    decryption and signature generation. \newline
    Adversary may act on time, timing how much time is spent on cryptographic operation, or how the device behave on fault injection.  
    \begin{figure}
        \centering
        \includegraphics[scale=0.7]{img/security_model_including_side_channel.png}
        \caption{Security model including side-channel.}
        \label{fig:sec_mod_inc_side_channels}
    \end{figure}
    
\end{itemize}

\subsection{Passive and active attacks}

Passive attacks are those that do not noticeably interfere with the operation of the target system. 

Active attacks when adversary exerts some influence on the behavior of
the target system. While the actively attacked system may or may not be
able to detect such influence, an outsider observer would notice a
difference in the operation of the system.

Attacks are divided into three categories: invasive attacks, semi-invasive attacks, and non-invasive attacks. 

\subsubsection{Invasive attacks}

Invasive attacks involves de-packaging to get direct access to the internal components of cryptographic models or devices.

For instance, a practical example could be attacking the passivation layer of a cryptographic module and place a probing needle on a data bus to see the data transfer. 

\subsubsection{Semi-invasive attacks}
In semi-invasive attacks, we access the devices without damaging the passivation layer or make electrical contact with unauthorized surface. 

\subsubsection{Non-invasive attacks}
Attack the device just by analyzing the information that are available from the external. 
One example could be the timing attack, that analyze the times spent by the device performing certain operation to deduce how long did it take to perform a certain operation. 
These types of attacks are completely undetectable, as the act from the external. 

\subsection{Most commonly known Side Channel Attacks}
\begin{itemize}
    \item Timing Attack
    \item Fault Attack
    \item Power Analysis Attack 
    \item EM Attack 
\end{itemize}

\subsubsection{Timing attack}
Timing attacks can intercept secret parameters from cryptographic algorithms due to a leak of information obtained during a careful statistical analysis. 

\subsubsection{Fault attack}
Fault attacks involve injecting a fault in cryptographic modules.
The fault is injected at the appropriate time during the process. 
More in particular, the fault act in voltage, clock, temperature, radiations, light, and other physical property. 

\subsubsection{Power Analysis Attack}
This passive side channel attack watches a trace or power usage of the system. 
Target can be an encryption key that is not readable.
Power usage is kept monitored and logged, power usage spikes are associated with possible bytes in a key.\newline

Simple power analysis: guess from power trace which particular instruction is being executed at a certain time (and possibly input and output values). Attacker needs an exact knowledge of the implementation.\newline

Differential power analysis: in this case attacker does not need to know the exact implementation to perform the attack. 

\subsection{Electo Magnetic Attack}
Electo magnetic attacks involve in examine the radiations emanated and understand their casual relationship with data computation. 

\subsection{Side-Channel Attacks countermeasures}

Possible solution to side-channel attacks can be: 
\begin{itemize}
    \item modify the program execution to interfere in the attacker analysis. It is sort of obfuscation, but referred to timing shifts, wait states, dummy instructions insertion, and operation execution randomization.
    \item replacing critical assembler instructions with ones whose "consumption signature" is hard to analyze.
    \item masking data and key with different random mask generated at each run.
\end{itemize}